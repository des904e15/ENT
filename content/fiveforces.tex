\subsection{Five forces}
 Porter præsenterer også en alternativ måde at forstå et startups succes, kaldet ``Five forces''.
 Disse ``five forces'' beskriver den konkurrencekontekst som virksomheden indgår i.
 Følgende beskrivelse er baseret på \citet[p.~16]{rose2012software}.

 Ifølge Porter er succesen af et startup delvis defineret af hvordan virksomheden konkurrerer med andre virksomheder på samme marked.
 Porters model kan ses på \cref{fiveforces}.

\begin{figure}[H]
	\includegraphics[width=\textwidth]{fiveforces.PNG}
	\caption{Five forces modellen}
	\label{fiveforces}
\end{figure}

\paragraph{Five Forces i SharePass' marked}
\label{par:five_forces_i_sharepass_marked}

For at beskrive det marked SharePass skal konkurrere på benyttes Five Forces modellen. 
Hver af de fem markedskræfter er beskrevet med de mest karakteriserende elementer:

\begin{itemize}
	\item \textbf{Rivalry amongst existing firms:} Der findes mange password managers, både som rent software, hardware og en kombination af de to. Typisk benyttes en abonnementsmodel.
	\item \textbf{Threat of new entrants:} Vores produkt har en væsentlig ``switching cost''\footnote{Det engelske udtryk ``switching cost'' er brugt i denne kontekst til at beskrive det en kunde skal ændre for at skifte fra et produkt til et andet.}  i forhold til eksisterende løsninger (der dog ikke er lige så integreret i virksomheden). 
	\item \textbf{Threat of substitute products:} Vores produkt kan få konkurrence af mindre avancerede løsninger der har lavere ``switching cost''.
	\item \textbf{Determinants of Supplier Power:} Vores hardware tager ikke udgangspunkt i en specifik leverandør og er derfor let at skifte. 
	\item \textbf{Determinants of Buyer Power:} Fokus er at skræddersy til virksomheder hvilket betyder at virksomhederne har en stor magt.
\end{itemize}
