\subsection{Kraaijenbrink}
Arbejdet fra \citet{kraaijenbrink2012nature} kritiserer de seks dimensioner brugt til at sammenligne causation og effectuation.
Nøglepunktet er at \citet{sarasvathy2001effectuation} bruger pragmatisme til at argumentere for effectuation modellen.
Men Kraaijenbrink mener det er bedre at den pragmatiske model bliver brugt når mennesker foretager handlinger.
Kraaijenbrink viser dette ved at gennemgå de seks dimensioner og vise hvordan de er blevet simplificeret og tilpasset causation og effectuation sammenlignings processen.
Vi vil kortfattet give en definition af pragmatisme og derefter gennemgå de seks dimensioner og vise hvad vi har brugt det Kraaijenbrink har fundet til.

\subsubsection{Pragmatisme}
Den pragmatiske model er defineret af følgende tre betegnelser.
\begin{itemize}
\item \textbf{Situatedness} - mennesker opfatter verdenen som en liste af muligheder som afgør hvordan de kan udføre ting.
\item \textbf{Corporeality} - opfattelsen er afgrænset af ens egne egenskaber og viden.
\item \textbf{Sociality} - social interaktion har indflydelse på menneskers handlinger.
\end{itemize}
En god beskrivelse a de tre betegnelser fra \citet[p.~195]{kraaijenbrink2012nature}:
\begin{quote}
  \textit{``First, humans always perceive the world in terms of the possible actions they can take. Hence, they perceive the world as a set of alternative opportunities that allow them to do certain things while being constrained from doing other things. Second, in perceiving these opportunities, humans are facilitated and constrained by their own body – their own capabilities, skills, and existing knowledge. Humans have a perception of their own abilities and take this into consideration when judging the opportunities they face. Finally, being social creatures, humans are facilitated and constrained by others. Humans mutually influence and persuade one another to take particular actions and to refrain from taking other actions.''}
\end{quote}

\subsubsection{De seks dimensioner}
En kort beskrivelse af hver dimension og Kraaijenbrinks kritikpunkter efterfulgt af hvad vi har brugt det til.

\paragraph{Starting point}
Denne dimension omhandler hvordan man starter ud i et projekt.
I Causation ser man på hvad man gerne vil opnå (Ends are given).
I Effectuation ser man på hvad man har til rådighed (Means are given).
Vi har kigget på hvad vi har til rådighed (se \cref{resource-based}) uden at det er direkte sammenlignelig med effectuation.

\paragraph{Assumptions on future}
Denne dimension omhandler hvordan man ser ud i fremtiden.
I Causation mener man at hvis man kan forudsige hvad der sker fx på markedet giver det kontrol.
I Effectuation er det omvendt, hvis man kan kontrollere fremtiden har man ikke brug for at forudsige den.
Kraaijenbrink mener at dette burde være to dimensioner og at en person der bruger causation eller effectuation kan vælge kontrol frem for forudsige eller omvendt.
Han mener at man her tager en pragmatisk beslutning.
Vi har valgt at kontrollere fremtiden ved at sætte en høj standard for sikkerheden i produktet, se \cref{paradigme}.

\paragraph{Predisposition toward risk}
Denne dimension omhandler hvordan man omgåes økonomiske risici.
I Causation gør man det ved at prøve at optimere og maksimere omsætningen.
I Effectuation ser man på hvor meget kapital de involverede er villige til at miste.
Her mener Kraaijenbrink igen at den ene model ikke udelukker den anden.
Dvs. man kan vælge hvad der giver mest mening pragmatisk set.

\paragraph{Appropriate for}
Denne dimension beskæftiger sig med om hvorvidt et firma fokuserer på nye eller eksisterende produkter og markeder.
Causation er for eksisterende produkter og markeder og effectuation for nye produkter og markeder.
Kraaijenbrinks første kritikpunkt er at det fremgår ikke klart i \citet{sarasvathy2001effectuation} om hvorfor man ikke skulle kunne anvende effectuation på eksisterende markeder.
Det andet kritikpunkt er at effectuation typisk kun er brugbart i agil udvikling.
Da vores produkt som sådan ikke er nyt, men nærmere en kombination af eksisterende løsninger kigger vi efter nye markeder, se \cref{paradigme}.

\paragraph{Attitude toward outside firms}
Denne dimension omhandler hvordan et firma begår sig i forhold til andre firmaer.
Enten samarbejder man (Effectuation) eller så er man i åben konkurrence (Causation).
Kraaijenbrink mener at der er mange firmaer og at litteraturen peger på at man sagtens kan have begge dele på en gang.
Vi har valgt at sammenarbejde se \cref{paradigme}.

\paragraph{Type of model}
Denne dimension omhandler hvilket miljø produktet skal udvikles til.
Det er om det kan udvikles med en lineær proces (statisk miljø) eller en cyklisk proces (dynamisk miljø).
Igen mener Kraaijenbrink at det ene ikke udelukker det andet.
Vi har valgt agil udvikling fordi produktet skal være meget brugervenligt og skal så hvidt muligt ikke ændre arbejdsfremgangen hos virksomheder.
