%!TEX root=../master.tex
\subsection{Pattern overvejelser}

\mikael{Overvejelser ud fra Osterwalder side 52-119.. R I P}
\ivan{how much?}
\mikael{Hvorfor er det ikke en af de 5?}

Ifølge \citet[pp. 52-119]{osterwalder2009business} er der 5 velkendte BM patterns; `Unbundling', `Long Tail', `Multi-Sided Platform', `FREE', `Open'.
Formålet med disse patterns er det samme som anvendelsen af patterns alle andre steder, som fx. indenfor software udvikling.

\subsubsection{Unbundling}
Der er 3 forskellige fokus forretninger kan have: Customer relations, Product Innovation og Infrastructure.
En forretning kan have alle 3 som fokus, men det kan give konflikter og kompromiser, hvorfor man ser helst kun at fokusere på én af de tre.
Dette er især gældende indenfor et større marked, hvor det er svært at være leder indenfor alle 3 fokus på én gang.

Dette er nok det nærmeste vi kommer de 5 patterns ift. vores forretning.
Vi har nævnt tidligere at vi gerne vil tilbyde særligt god kundesupport.
Der er en vis produktinnovation, da vi vil kombinere eksisterende produkter i en ny form for løsning, samt tilbyde dem til et nyt marked.
Vi har dog ingen større, eller behov for, infrastruktur.

\subsubsection{Long Tail}
Her er fokuset på at tilbyde mange lavt-sælgende niche-produkter, sådan at det samlede antal salg er højt.
Det kræver en god platform/infrastruktur at gøre mange produkter tilgængelig uden stor overhead.

Dette er det modsatte af hvad vores forretning prøver at formå.
Vi har overordnet set et enkelt produkt og ingen større platform til distribuering af produktet.

\subsubsection{Multi-Sided Platform}
Forretninger der ønsker at skabe en multi-sided platform er afhængigt af flere bruger-grupper for at skabe værdi.
Platformen gør det muligt for nogle at skabe indhold, og samtidig gør det muligt for andre bruge dette indhold.
Det kan være muligt at det er mere komplekst, med flere grupper.
Det er også muligt at brugere tilhører flere grupper, så de både skaber og benytter indhold.

Denne er svær at relatere til vores forretning.
Selvom vi har et produkt hvor nogle kan skabe indhold (passwords og licenser) og andre skal bruge disse, giver det ikke mening at tilknytte betaling på de enkelte `transaktioner'.

\subsubsection{FREE}
Der er 3 sub-mønstre for FREE BMs: Advertising-based, Freemium og `Bait \& Hook'.
Advertising-based, er baseret på Multi-Sided Platform.
Freemium tilbyder gratis `grund-pakke' med mulighed for at tilkøbe premium ydelser.
`Bait \& Hook' tilbyder initielt et (næsten) gratis produkt, hvorefter kunder forventes at lave tilkøb.
Ens for de to sidste er at de kan tabe penge på grund-pakken/det initielle produkt, så længe nok kunder tilkøber efterfølgende.
Forskellen er at Freemium er fint nok for de fleste og Premium-tilkøbet er ikke nødvendigt.
`Bait \& Hook' derimod forventer tilkøb, enten i form af mange små eller få store tilkøb.

Hvis vi skulle lave en grundpakke, eller initierende produkt, for at følge dette mønster, ville dette produkt være for dyrt og derved risikoen for høj.

\subsubsection{Open}
Der er 2 typer af Open BMs: `Inside-Out' og `Outside-In'.
Ved Inside-Out sælger man idéer, i form af intellektuelt eller fysisk produkt.
Ved Outside-In udnytter man andres idéer.

Det kunne forestilles at vi skabte en platform som kunne anvendes til andre områder, således vi på sigt kunne agere `Outside-In' og sælge denne platform.
Vi kunne også være interesseret i partnerskab med `Inside-Out' forretninger, således de enkelte dele af vores produkt er gode.

\subsubsection{Vores BM ift. de 5 patterns}
\mikael{Tror lige vi skal snakke om denne her.}
