%!TEX root = ../master.tex
\subsection{Faser i Business Model Design Processen}
Ifølge \citet[pp. 244-259]{osterwalder2009business} er der et forslag til en anvendelig proces til når der skal udarbejdes en ny BM.
Her beskrives 5 faser: Mobilize, Understand, Design, Implement og Manage -- hvor hver fase består af en række aktiviteter der kan udføres for at opnå målene for de enkelte faser.
Undervejs vil der blive både udarbejdet og indført en BM, som til sidst vil blive vedligeholdt.
Den overordnede idé med dette er at det er svært at designe en BM, men lige så snart dette er gjort er det nemt at udføre denne.
Dette er en design-orienteret tilgang, i modsætning til en beslutnings-orienteret tilgang som derimod siger at det er nemt at finde på idéer og problemet ligger derimod i at vælge en af disse.

For at lave en (ny) BM for vores forretning, har vi fulgt de fem faser fra \citet[pp. 244-259]{osterwalder2009business}.
For hver fase har vi valgt de aktiviteter vi mener er relevante at nævne ifm. udviklingen af vores BM.

\subsubsection{Mobilize}
\paragraph{Test preliminary business ideas}
Vores BM diskuteres internt, på tværs af hele organisationen, da denne ikke er så stor.
Det er vigtigt at udviklere, sikkerhedsekspert, usability-ekspert og sælgere har en fælles forståelse og enighed omkring forretningens BM.

\paragraph{Assemble team}
Som blev nævnt i \cref{value_chain} og \cref{resource-based} er der samlet et hold.
Dette hold er alsidigt, således at en grundig evaluering af de initielle BMs kan foretages.
Her er det muligt for ethvert medlem af forretningen at bruge deres område/ekspertise til at evaluere de enkelte dele af forretningens BM.

\paragraph{Find lokaler}
For at kunne demonstrere produktet ordentligt til kunder er det nødvendigt at kunne fremvise produktet i en kontekst der tilnærmer sig virkeligheden.
Det er derfor vigtigt at finde nogle lokaler der tillader at vi kan sætte en sådan demonstration op.

\subsubsection{Understanding}
\paragraph{Study potential customers}
Vi vil undersøge vores potentielle kunder (virksomheder) og den måde hvorpå de håndtere sikkerheds nu.
Altså, hvilke sikkerhedsprotokoller har de, hvis nogen, og i så fald hvordan indkorporeres de i deres arbejdsproces.
Det er også vigtigt at vi fra starten af virker professionelle, så vi kan udvinde en tillid, som er vigtigt med så følsomt et område.

\paragraph{Research already tried}
Her kunne vi kigge på eksisterende løsninger (e.g. LastPass, 1Password) og se hvad de har gjort for at blive udbredte indenfor et marked der minder om vores eget.

\subsubsection{Design}
\paragraph{Prototype}
Vi forestiller os et usability lab i vores egen bygning, som skal afspejle den løsning vi vil give vores kunder.
Det vil sige vi har brug for et eller flere rum hvor vores løsning skal bruges for at få adgang til rummene.
I de rum skal også være det system der skal bruges til at logge på, samt håndtere centralisering og deling af passwords.

\paragraph{Brainstormet og testet BMs}
For at få testet vores BMs udefra vil vi konsultere os med dem vi mener der ved mest indenfor skabelsen af BMs: Borean Innovation, SEA eller en anden venture capital investor).

\subsubsection{Implement}
\paragraph{Communicate and involve}
Alle er allerede involveret, derfor behøver vi ikke at kommunikere og involvere andre. \stefan{expand eller slet}

\subsubsection{Manage}
Her gælder samme holdning som kan ses tidligere i denne sektion.\bruno{Hvorhenne? :)} \stefan{expand eller slet}
