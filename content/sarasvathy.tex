%!TEX root=../master.tex

Dette afsnit omhandler \citet{sarasvathy2001effectuation}.

Sarasvathy foreslår et alternativ til den klassiske kausale tankegang omkring dannelse af nye virksomheder.
Hvor man tidligere har haft et fokus på at forstå sine omgivelser, for bedre at kunne forudsige og reagere på ændringer, foreslår Sarasvathy \textit{effectuation}.
Effectuation går ud på at styre sine omgivelser og tage udgangspunkt i de resourcer man allerede har til rådighed.

Der findes 4 principper i Effectuation:
\begin{itemize}
  \item Affordable loss -- rather than expected returns
  \item Strategic alliances -- rather than competitive analyses
  \item Exploitation of contingencies -- rather than preexisting knowledge
  \item Control of an unpredictable future -- rather than prediction of an uncertain one
\end{itemize}

\subsubsection{Relation til vores forretning}
Sarasvathy sætter de to paradigmer, causation og effectuation, op ret firkantet.
Ikke nødvendigvis at en er bedre end den anden, men at de tjener hver deres formål afhængigt af hvilken type forretning man ønsker at starte/har startet.
I tilfælde af at man prøve at overtage et marked der allerede eksisterer eller tilbyde et produkt der allerede findes, passer causation bedst, da der allerede er megen viden tilgængelig, således man kan analysere sig frem til en god indgangsvinkel.
Hvis man derimod har et nyt produkt eller prøver at komme ind på et nyt markedssegment, passer effectuation bedre, da det er mindre begrænsende.
Som man ser i eksemplet med U-Haul er der risiko for at forretningen slet ikke kommer fra start, da det kan vurderes ud fra tidlig analyse at det ikke er bæredygtigt at starte forretningen.
Derfor bør man se på hvilke resourcer man har og gå på kompromis med de mål man vil nå, for at komme i gang.

Hvis man ser på vores valg af paradigme i \cref{paradigme} hælder vi mest til effectuation.
Den største forskel er at vi har valgt at gå med venture capital, da vi mener at det vil være svært at få et start produkt i høj nok kvalitet til at vi kan gå videre.
Hvis vi havde valgt en ren effectuation tilgang, havde vi enten startet med et tilpas småt produkt som vi kunne have dækket gennem egen finansiering, eller ved at have holdt det som et side-projekt til et andet job (evt. studie).
Vi kunne også have valgt at skrue ned for forventningerne og have tilbudt et andet produkt end det vi vil kalde den endelige SharePass løsning, men måske i stedet starte med at tilbyde andre services, såsom sikkerheds-konsultering, indtil vi har de nødvendige resourcer til den løsning vi gerne så.
