\subsection{Design - Storytelling}
Dette afsnit præsenterer den teknik vi har benyttet i forbindelse med designet af vores business model - storytelling.
Beskrivelsen af storytelling er baseret på \citet[p.~125]{osterwalder2009business}
Storytelling går ud på at undersøge forskellige business models igennem en historie der beskriver hvordan produktet skal fungere.
Denne historie kan præsenteres på flere måder.
Man kan bruge illustrationer som udgangspunkt til en diskussion om produktet, man kan lave en videopræsentation der præsenterer hvordan produktet skal fungere eller man kan lave en tegneserie der beskriver hvordan interaktion med produktet skal forløbe.
Storytelling har til formål at gøre et koncept mere håndgribeligt så det er lettere at diskutere og derved lettere at udvikle sin business model.

Vi har valgt at lave en tegneserier der beskriver først den nuværende situation og derefter hvordan vi forestiller os at forbedre situationen med vores produkt.
I \cref{story:badpass:boss} og \cref{story:sharing} præsenteres de potentielle problemstillinger der afhjælpes ved brug af SharePass.

I \cref{story:badpass} anvender en virksomheds medarbejdere dårlige\footnote{Eksempelvis meget korte eller almindeligt brugte adgangskoder.} adgangskoder.
I \cref{story:sharing} deles adgangskoder (eksempelvis for nye medarbejdere) via email.
Alternativt ville denne deling også kunne ske på papir ved at ophænge \emph{fælles} adgangskoder i kontorer.

Problemerne afhjælpes i \cref{story:sharepass} med SharePass.
Her får virksomhedens direktør et system der løser ovenstående problemer ved automatisk at generere nye adgangskoder for systemets medarbejdere og lade disse blive anvendt ved hjælp af elektroniske identifikations-kort.

{
% Command and settings for creating the storytelling comics
\newcommand{\storyfig}[1]{\fbox{\includegraphics[height=5cm]{graphics/story_#1}}}
\setlength\fboxsep{0pt}
\setlength{\fboxrule}{1pt}

\begin{sidewaysfigure}
\centering
\subfloat[En virksomhed bestående af direktør Steen og hans to ansatte Anders og Birte.]
{\storyfig{01}\label{story:badpass:init}} \quad
\subfloat[Anders og Birte tilgår services ved hjælp af deres adgangskoder \emph{abc} og \emph{1234}.]
{\storyfig{03}\label{story:badpass:pass}} \quad
\subfloat[Steen er bekymret for beskyttelsen af virksomhedens oplysninger på grund af medarbejdernes valg af koder.]
{\storyfig{04}\label{story:badpass:boss}}
\caption{Medarbejdere anvender dårlige adgangskoder}\label{story:badpass}
\end{sidewaysfigure}

\begin{sidewaysfigure}
\centering
\subfloat[En virksomhed bestående af direktør Steen og hans to ansatte Anders og Birte.]
{\storyfig{01}\label{story:sharing:init}} \quad
\subfloat[Anders mangler adgang til en applikation og Birte sender derfor den fælles adgangskode til Anders via email.]{\storyfig{07}\label{story:sharing:pass}} \quad
\subfloat[Steen er bekymret for beskyttelsen af virksomhedens adgangskoder på grund af medarbejdernes delings-metode.]
{\storyfig{05}\label{story:sharing:boss}}
\caption{Deling af adgangskoder foregår åbent og ukrypteret.}\label{story:sharing}
\end{sidewaysfigure}

\begin{sidewaysfigure}
\centering
\subfloat[Anders og Birte får hver udleveret et personligt identifikations-kort med en NFC chip.]
{\storyfig{08}\label{story:sharepass:nfc}} \quad
\subfloat[De koder Anders og Birte anvender er ikke længere selvbestemt.
Der anvendes derfor længere og ikke almindelige adgangskoder.]
{\storyfig{09}\label{story:sharepass:login}}\\
\subfloat[Adgangskoder administreres af en central server og deles elektronisk med medarbejdernes kort.
Ændring af adgangskoder og tildeling af licenser/koder til en medarbejder administrerers af det centrale system.]
{\storyfig{11}\label{story:sharepass:server}} \quad
\subfloat[Steen er tilfreds da han nu har mulighed for, via det centrale system, at administrere Anders og Birtes licenser og er sikret at adgangskoderne er lange.]
{\storyfig{10}\label{story:sharepass:success}}
\caption{Brug af SharePass sikrer at Steens medarbejdere anvender bedre\\ adgangskoder og administrerer deling blandt medarbejderne.}\label{story:sharepass}
\end{sidewaysfigure}
}