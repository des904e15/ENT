%!TEX root = ../master.tex
\subsection{Karakteriser og diskuter value proposition}

Denne sektion vil diskutere den value proposition der blev præsenteret i business modellen på \cref{bm} i et bredt perspektiv.
Diskussionen vil tage udgangspunkt i koncepterne præsenteret i \citet[p.~40]{rose2012software}.

\subsubsection{Type af entreprenør}
Giarratanas tre typer af entreprenør er beskrevet som følger

\begin{description}
	\item[Innovator] Skaber nye produkter der ikke er set før
	\item [Arbitrageur] Udnytter ineffektivitet i markedet. For eksempel kan nuværende løsninger være for dyre eller for besværlige for en bestemt kundegruppe
	\item [Coordinator] Udnytter ressourcer på en alternativ måde
\end{description}

Af disse typer er vores virksomhed en arbitrageur.
Vi bringer noget nyt til markedet ved at kombinere kendte teknologier (password manager, hardware nøgler)på en måde der eliminerer ineffektive arbejdsgange i virksomheder der muligvis allerede bruger de produkter vi kombinerer.

\subsubsection{Competitor orientation}
Ifølge Mueller \citep[p.~40]{rose2012software} er der en positiv korrelation mellem succesfulde startups og startups der har en retning i forhold til konkurrencen (competitor orientation).
Der findes følgende strategier:
\begin{description}
	\item[Leader] Image, innovation, kobler teknologi men brugerbehov
	\item[Follower] Optimerer features på eksisterende produkter, design
	\item [Exploiter] Optimerer kvalitet, modifikationer af eksisterende produkter
	\item [Extender] Prisoptimering, fokus på salg
\end{description}

Med disse hører følgende business strategier:
\begin{description}
	\item[Cost leadership] Billigere produkter en konkurrencen
	\item [Differentiation] Produkter der er unikke og skiller sig ud
	\item [Niche] Konkurrerer i et område med få konkurrenter
	\item [Speed to market] Evnen til at innovere og sende nyskabelser på markedet hurtigst.
\end{description}

Vores virksomhed forsøger at være leader indenfor den del af markedet vi åbner for.
Det er derfor vigtigt at vi fokuserer på at have et godt image.
Dette vil vi sørge for ved at tilbyde direkte kundesupport og udvikle et sikkert og brugervenligt system.

Af de fire business strategier er vore virksomhed i en niche hvor vi konkurrerer med nogle få specialiserede konkurrenter.
Det er vigtigt for virkomheden at have vægt på differentiation, da vi er nødt til at skille os ud fra de etablerede virksomheder for at overbevise kunder om at vores løsning, der typisk vil være mere omfattende end konkurrentens, er værd at implementere.

\subsubsection{Entry and growth}
Ifølge Porter er der altid "barriers of entry" når man begiver sig ind på et allerede eksisterende marked \citep[p.~50]{rose2012software}.
Naturligvis vil eksisterende virksomheder gøre alt for at man som ny på markedet ikke tager deres kunder.
Desuden har man som startup mange ulemper der giver de etablerede virksomheder fordelen: få midler, få ansatte med mange opgaver, intet omdømme osv.
Ojala og Tyrväinen har opsat følgende drivkræfter for et startups succes i at begive sig ind på markedet: \citep[p.~50]{rose2012software}

\begin{description}
	\item[Innovation] Software produktet, algoritmer og teknikker
	\item [Capabilities of entrepreneurs] Både specialiserede tekniske evner og evner som entreprenør
	\item [Patents] Beskyttelse af intellektuel ejendom
	\item [Specialization in a specific market niche] 
\end{description}

For vores virksomhed er det vigtigt at have vægt på alle fire drivkræfter, men især fokus på innovation af produktet og specialisering i nichen er vigtigt for at det skal blive en succes.
Det er afgørende at vi skiller os ud fra de etablerede virksomheder og udvikler denne differentiering til vores fordel.