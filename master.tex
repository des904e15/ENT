\documentclass[a4paper]{article}

\usepackage[english]{babel}
\usepackage[utf8]{inputenc}
\usepackage{amsmath}
\usepackage{graphicx}
\usepackage[colorinlistoftodos]{todonotes}
\usepackage[backend=bibtex,
  bibencoding=utf8,natbib
  ]{biblatex}
\addbibresource{bib}

\title{Your Paper}

\author{You}

\date{\today}

\begin{document}
\maketitle

\todo{STEFAN: husk at indsætte refs!}
\section{Motivation}
\subsection{Storytelling - en comic strip}

\section{Idé}
\todo{Overvej fysisk login portal!}
\todo{Valgte teorier: Value chain, five forces, resources-based theory}

Password manager med deling for firmaer og private

En form for hardware der lagrer krypterede passwords fx en USB nøgle med indbygget software der sørger for at anvende de rigtige koder.
Lokal central der indeholder alle passwords, som nøgle synkroniseres fra.
Administrering af egne nøgler, samt deling af disse.
Automatisk opdatering af intra-net systemers nøgler (tænk nyt password hver måned).

\subsection{Value chain}
Taget fra \citet[p.~12]{rose2012software}.
\subsubsection*{Primary activities}
BUY:
\begin{itemize}
\item NFC kort
\item Fysisk login med NFC kort hos firma - her opdateres passwords på kortet
\end{itemize}
ADD VALUE:
\begin{itemize}
\item Sikkerhedsprotokol mellem fysisk login portal og server
\item Sikkerhedsprotokol mellem klient og NFC kort
\item Sikkerhedsprotokol mellem fysisk login portal og NFC kort
\item Software til password server
\end{itemize}
SELL:
\begin{itemize}
\item Sælg bedste produkt i verden
\item Tjen en masse penge på support
\end{itemize}

\subsubsection*{Support activities}
SELL
\begin{itemize}
\item Marketing
\item Hyre - sikkerhedsekspert, hardwaredesigner
\end{itemize}

\subsection{Resource-based theory}
Taget fra \citet[p.~12]{rose2012software}.
\paragraph{Ressourcer:}
\begin{itemize}
\item Team:
  \begin{itemize}
  \item Programmør
  \item Sikkerhedsekspert
  \item Hardwaredesigner
  \item Sælger
  \item Systemarkitekt
  \end{itemize}
\item Produkter
  \begin{itemize}
  \item NFC kort
  \item Fysisk login portal
  \item Software klient - ansattes pc kommunikation med NFC kort
  \item Server sofware - deling og håndtering af passwords og licenser
  \item Sikkerhedsprotokol mellem fysisk login portal og server
  \item Sikkerhedsprotokol mellem klient og NFC kort
  \item Sikkerhedsprotokol mellem fysisk login portal og NFC kort
  \end{itemize}
\item Pakkeløsning
\item Kundeservice
  \begin{itemize}
  \item Opsætning
  \item Udvidelse
  \item Support
  \end{itemize}
\item Viden om:
  \begin{itemize}
  \item Sikkerhed
  \item Kunders(virksomheder) workflow
  \item Softwarearkitektur
  \end{itemize}
\item Demo setup - som kan bruges til fremføring og test
\end{itemize}
\paragraph{Beskrivelse af team}
\todo{Ivan hvad mener du med: 'Describe your technological expertise'}
\paragraph{Formåen:}
Levere en samlet løsning, skræddersyet til kunden som er sikker at bruge.
En sikker løsning for kunden og hans ansatte.
De ansatte skal ikke tænke på sikkerhed men data er alligevel beskyttet af sikre passwords.

\section{Business model}
\todo{Se på requirements}

\subsection{Discussion}
\paragraph{How does our BM support our implementation?}
\todo{Hvad skal der være her? Vi har jo lavet det ud fra at vi kan implementere vores produkt.}
\paragraph{How does the canvas model change over time?}
Lige nu er der meget fokus på sikkerhed så det er godt at hoppe ind på markedet nu - det kan være at dette fokus ændre sig.
Det ligger op til at undersøge nye markeder.
En mulighed er at udvide til det private marked, dvs. at customer segments bliver udvidet.
En anden mulighed er at tilbyde flere services:
\begin{itemize}
\item Tilbyde forskellige slags hardware løsninger til at opbevare passwords(i stedet for kun NFC).
\item Tilbyde en server til at distribuere passwords til kunder.
\end{itemize}
Hvis disse to services tilføjes ændres value proposition så vores løsning kan skræddersyes mere til kunden.
Key activities og key resources ændres da der nu skal tages højde for at udvikle og vedligeholde de nye services.
For at spare penge som virksomhed eller gøre det mere tilgængeligt for kunder med færre midler kunne man ændre after sales til at foregå over en webside.
Dette ændre revenue streams til at der tilbydes billigere support eller kun en slags support over webside.
Det ændrer også customer segments fordi det giver adgang til firmaer og private med færre midler.

\paragraph{Pattern overvejelser}
\todo{Overvejelser ud fra Osterwalder side 52-119.. R I P}
\todo{Ivan - how much?}

\paragraph{Design tilgang}
Storytelling måske ingen ved det.......wat wat wat

\paragraph{Strategy concerns}
\todo{Læs strategy oste-bogen: 192-231}

\section{Analyse, design and act(ADA) eller consider, do and adjust(CDA)}
\todo{Jeremys bog, side 27-38}

\printbibliography[heading=bibintoc]

\end{document}
