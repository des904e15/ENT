\documentclass[a4paper]{article}

\usepackage[english]{babel}
\usepackage[utf8]{inputenc}
\usepackage{amsmath}
\usepackage{graphicx}
\usepackage[colorinlistoftodos]{todonotes}
\usepackage[backend=bibtex,
  bibencoding=utf8,natbib
  ]{biblatex}
\addbibresource{bib}

\title{SharePass - Sikre passwords og sikker deling af passwords og licenser for virksomheder}

\author{Stefan Marstrand Getreuer Micheelsen, Mikael Elkiær Christensen, Mikkel Sandø Larsen og Bruno Thalmann}

\date{\today}

\begin{document}
\maketitle

\todo{STEFAN: husk at indsætte refs!}
\todo{MIKAEL: kort teoretisk opsummering når det bruges}

\section{Idé}
\todo{Overvej fysisk login portal!}
Password manager med deling for firmaer og private

En form for hardware der lagrer krypterede passwords fx en USB nøgle med indbygget software der sørger for at anvende de rigtige koder.
Lokal central der indeholder alle passwords, som nøgle synkroniseres fra.
Administrering af egne nøgler, samt deling af disse.
Automatisk opdatering af intra-net systemers nøgler (tænk nyt password hver måned).

\subsection{Storytelling - en comic strip}

\subsection{Value chain}
Taget fra \citet[p.~12]{rose2012software}.
\subsubsection*{Primary activities}
BUY:
\begin{itemize}
\item NFC kort
\item Fysisk login med NFC kort hos firma - her opdateres passwords på kortet
\end{itemize}
ADD VALUE:
\begin{itemize}
\item Sikkerhedsprotokol mellem fysisk login portal og server
\item Sikkerhedsprotokol mellem klient og NFC kort
\item Sikkerhedsprotokol mellem fysisk login portal og NFC kort
\item Software til password server
\end{itemize}
SELL:
\begin{itemize}
\item Sælg bedste produkt i verden
\item Tjen en masse penge på support
\end{itemize}

\subsubsection*{Support activities}
SELL
\begin{itemize}
\item Marketing
\item Hyre - sikkerhedsekspert, hardwaredesigner
\end{itemize}

\subsection{Resource-based theory}
Taget fra \citet[p.~12]{rose2012software}.
\paragraph{Ressourcer:}
\begin{itemize}
\item Team:
  \begin{itemize}
  \item Programmør
  \item Sikkerhedsekspert
  \item Hardwaredesigner
  \item Sælger
  \item Systemarkitekt
  \end{itemize}
\item Produkter
  \begin{itemize}
  \item NFC kort
  \item Fysisk login portal
  \item Software klient - ansattes pc kommunikation med NFC kort
  \item Server sofware - deling og håndtering af passwords og licenser
  \item Sikkerhedsprotokol mellem fysisk login portal og server
  \item Sikkerhedsprotokol mellem klient og NFC kort
  \item Sikkerhedsprotokol mellem fysisk login portal og NFC kort
  \end{itemize}
\item Pakkeløsning
\item Kundeservice
  \begin{itemize}
  \item Opsætning
  \item Udvidelse
  \item Support
  \end{itemize}
\item Viden om:
  \begin{itemize}
  \item Sikkerhed
  \item Kunders(virksomheder) workflow
  \item Softwarearkitektur
  \end{itemize}
\item Demo setup - som kan bruges til fremføring og test
\end{itemize}
\paragraph{Beskrivelse af team}
\todo{Ivan hvad mener du med: 'Describe your technological expertise'}
\todo{Svar: Udgangspunkt i os selv, men med hyring af folk udefra hvis nødvendigt.}
\paragraph{Formåen:}
Levere en samlet løsning, skræddersyet til kunden som er sikker at bruge.
En sikker løsning for kunden og hans ansatte.
De ansatte skal ikke tænke på sikkerhed men data er alligevel beskyttet af sikre passwords.

\paragraph{Five forces}
\todo{jeremys bog}

\section{Business model}

\subsection{Discussion}
\paragraph{Karakteriser og diskuter value proposition}
\todo{Jeremy s. 40-70}
\paragraph{How does our BM support our implementation?}
\todo{IVAN? Hvad skal der være her? Vi har jo lavet det ud fra at vi kan implementere vores produkt.}
\todo{Tænk på big-bang, MVP og bootstrapping.}
\paragraph{How does the canvas model change over time?}
Lige nu er der meget fokus på sikkerhed så det er godt at hoppe ind på markedet nu - det kan være at dette fokus ændre sig.
Det ligger op til at undersøge nye markeder.
En mulighed er at udvide til det private marked, dvs. at customer segments bliver udvidet.
En anden mulighed er at tilbyde flere services:
\begin{itemize}
\item Tilbyde forskellige slags hardware løsninger til at opbevare passwords(i stedet for kun NFC).
\item Tilbyde en server til at distribuere passwords til kunder.
\end{itemize}
Hvis disse to services tilføjes ændres value proposition så vores løsning kan skræddersyes mere til kunden.
Key activities og key resources ændres da der nu skal tages højde for at udvikle og vedligeholde de nye services.
For at spare penge som virksomhed eller gøre det mere tilgængeligt for kunder med færre midler kunne man ændre after sales til at foregå over en webside.
Dette ændre revenue streams til at der tilbydes billigere support eller kun en slags support over webside.
Det ændrer også customer segments fordi det giver adgang til firmaer og private med færre midler.

\paragraph{Pattern overvejelser}
\todo{Overvejelser ud fra Osterwalder side 52-119.. R I P}
\todo{Ivan - how much?}
\todo{Hvorfor er det ikke en af de 5?}

\paragraph{Design tilgang}
Storytelling måske ingen ved det.......wat wat wat

\paragraph{Strategy concerns}
\todo{Læs strategy oste-bogen: 192-231}

\section{Analyse, design and enact(ADE) eller consider, do and adjust(CDA)}
\todo{Jeremys bog, side 27-38}
Baseret på tabel s. 38.
\paragraph{Process:} Primært sekventiel, vi har hardware samt meget sikkerhedskritisk software, så det kunne være katastrofalt at få rullet et usikkert produkt ud for hurtigt.

\paragraph{Understanding of future:} Vi vil gerne styre fremtiden, altså sætte en høj standard for sikkerhed som andre vil få svært ved at følge.
Usability og at det passer ind i kundens/brugernes proces er vigtigt, så her skal vi være gode til at tilpasse os så vi kan få solgt vores produkt, altså så folk ikke ser det som en for stor hindring.

\paragraph{Attitude to market:} Forsøg på at skabe et nyt marked, ved at bringe noget som folk ikke er klar over de har brug for.
Selvom det som sådan ikke er et nyt produkt.

\paragraph{Attitude to technology development:} Vi vil gerne være foran de andre så vi kan sætte standarden for hvad god sikkerhed, så vi er de eneste der kan levere det optimalt.

\paragraph{Role of business planning:} Vi har ikke råd til fejl her, da vist vi først får leveret en dårlig løsning, vil folk få svært ved at have tillid til os efterfølgende.

\paragraph{Software development:} Agile og lean, da vi gerne vil have et velafprøvet og veltestet produkt.
Usability er igen vigtigt, da vi gerne vil have vores produkt så seamless integreret som muligt.

\paragraph{Attitude to change:} Nyt marked, så vi ser ændringer som muligheder for at lave et bedre og mere fleksibelt produkt.

\paragraph{Funding approach:} Venture capital så vi kan få sikret kvalitet inden produktet frigives, sådan at vi ikke mister credibility ved en for dårlig tidlig løsning.

\paragraph{Approach to others working in the same areas:} Samarbejde er godt.
Hvis vi kan undgå at opfinde/lave alting selv er det en fordel.
Så kan vi udnytte andres erfaringer, merge idéer og sælge dem videre som pakkeløsning.

\paragraph{Approach to intellectual property:} Platformen er lukket, men de enkelte dele er åbne.

\paragraph{Partnering and Networking:} Netværk og samarbejde er godt, samme som ovenover.

\paragraph{Time to market:} Se 'Funding approach'.

\paragraph{Phases or activities in the business model process}
\todo{Oste-bogen, side 244-?}
\paragraph{Financial concerns relativ to the choice of paradigm}
\todo{On hold}

\section{Diskuter to papers}
\subsection{Er vores valg a paradigme teoretisk eller empirisk validt?}
\subsection{Relater concepter brugt til artiklerne fra kurset}

\printbibliography[heading=bibintoc]

\end{document}
