\documentclass[a4paper]{article}

\usepackage[english]{babel}
\usepackage[utf8]{inputenc}
\usepackage{amsmath}
\usepackage{graphicx}
\graphicspath{{./graphics/}} 
\usepackage[colorinlistoftodos]{todonotes}
\input{todonotes}
\usepackage[backend=bibtex,
  bibencoding=utf8,natbib
  ]{biblatex}
\addbibresource{bib}
\usepackage{cleveref}


\title{SharePass - Sikre passwords og sikker deling af passwords og licenser for virksomheder}

\author{Stefan Marstrand Getreuer Micheelsen \and Mikael Elkiær Christensen \and Mikkel Sandø Larsen \and Bruno Thalmann}

\date{\today}

\begin{document}
\maketitle
\listoftodos

\bruno{Skal vi tilføje kontor og et kontor/rum til demo-setup som en ressource og evt. en aktivitet i mobilize som hedder at finde det rigtige faciliteter?}
\stefan{husk at indsætte refs!}
\mikael{kort teoretisk opsummering når det bruges}

\section{Idé}
\bruno{Overvej fysisk login portal!}
Password manager med deling for firmaer og private

En form for hardware der lagrer krypterede passwords fx en USB nøgle med indbygget software der sørger for at anvende de rigtige koder.
Lokal central der indeholder alle passwords, som nøgle synkroniseres fra.
Administrering af egne nøgler, samt deling af disse.
Automatisk opdatering af intra-net systemers nøgler (tænk nyt password hver måned).

\subsection{Storytelling - en comic strip}

\subsection{Value chain}


Porter's value chain model beskriver et firma ud fra de aktiviteter der skal udføres for at tilføje værdi til de "rå" materialer.
Eftersom modellen er konstrueret med konventionelle fremstillingsvirksomheder i tankerne kan det ikke direkte overføres til softwarevirksomheder.
Softwarevirksomheder inkøber sjældent rå materialer i traditionel forstand, men ideerne gælder stadig på et konceptuelt niveau.

Følgendende beskrivelse er baseret på \citet[p.~12]{rose2012software}.

Aktiviterer inddeles i to kategorier, de primære og support aktiviteter.
De primære aktiviteter beskriver hvordan virksomheden omdanner rå materialer til det færdige produkt på hylden ved at tilføje værdi.
Modellen er illustreret på  \cref{valuechain}.

\begin{figure}
	\includegraphics[width=\textwidth]{valuechain.png}
	\caption{Porters valuechain model}
	\label{valuechain}
\end{figure}

\paragraph{Application}
We have used 

\subsubsection*{Primary activities}
BUY:
\begin{itemize}
\item NFC kort
\item Fysisk login med NFC kort hos firma - her opdateres passwords på kortet
\end{itemize}
ADD VALUE:
\begin{itemize}
\item Sikkerhedsprotokol mellem fysisk login portal og server
\item Sikkerhedsprotokol mellem klient og NFC kort
\item Sikkerhedsprotokol mellem fysisk login portal og NFC kort
\item Software til password server
\end{itemize}
SELL:
\begin{itemize}
\item Sælg bedste produkt i verden
\item Tjen en masse penge på support
\end{itemize}

\subsubsection*{Support activities}
SELL
\begin{itemize}
\item Marketing
\item Hyre - sikkerhedsekspert, hardwaredesigner
\end{itemize}

\subsection{Resource-based theory}
%!TEX root=../master.tex
\subsection{Resource-based theory}\label{resource-based}

Denne teori beskæftiger sig med hvad et firma formår på baggrund af dets ressourcer, og beskrives i \citet[p.~13]{rose2012software}.
Ressource i denne kontekst er et vidt begreb, hvilket kan ses af \cref{rbt_fig}.
Her fremgår ligeledes det mulige formåen.

\begin{figure}[H]
  \begin{center}
    \includegraphics[width=.8\textwidth]{resource_based_theory.png}
  \end{center}
  \caption{Et eksempel på ressourcer og mulig formåen i Resource Based Theory.}
  \label{rbt_fig}
\end{figure}

\paragraph{Ressourcer:}
Herunder følger en fortegnelse over de ressourcer vores virksomhed har til rådlighed:
\begin{itemize}
\item Team:
  \begin{itemize}
  \item Programmør
  \item Sikkerhedsekspert (konsulent)
  \item Hardwaredesigner
  \item Sælger
  \item Systemarkitekt
  \item Usability (konsulentfirma)
  \end{itemize}
\item Produkter
  \begin{itemize}
  \item NFC kort
  \item Fysisk login portal
  \item Software klient -- ansattes pcs kommunikation med NFC kortet
  \item Server sofware -- deling og håndtering af passwords og licenser
  \item Sikkerhedsprotokol mellem fysisk login portal og server
  \item Sikkerhedsprotokol mellem klient og NFC kort
  \item Sikkerhedsprotokol mellem fysisk login portal og NFC kort
  \end{itemize}
\item Pakkeløsning
\item Kundeservice
  \begin{itemize}
  \item Opsætning
  \item Udvidelse
  \item Support
  \end{itemize}
\item Viden om:
  \begin{itemize}
  \item Sikkerhed
  \item Kunders (virksomheder) workflow
  \item Softwarearkitektur
  \end{itemize}
\item Demo setup - som kan bruges til fremføring og test
\end{itemize}
\paragraph{Beskrivelse af team}
Vores team består af os selv, fire mand som står for:
\begin{itemize}
\item Systemarkitektur(hardware og software)
\item Til dels sikkerhed
\item Software
\item Support
\end{itemize}
Udover det består det af en sælger, en hardwaredesigner, en sikkerhedskonsulent og et konsulentfirma til evaluering af usability af produktet.

\paragraph{Formåen:}
Det giver os en mulighed for at levere en samlet løsning, skræddersyet til kunden som samtidig er sikker at bruge.
Det er en løsning hvormed der satses på god usability, og at det ikke forstyrrer de ansatte i deres daglidag, ved fx at ændre deres workflow.
De ansatte skal ikke tænke på sikkerhed men data er alligevel beskyttet af sikre passwords.


\subsection{Five forces}
 Porter præsenterer også en alternativ måde at forstå et startups succes, kaldet ``Five forces''.
 Disse ``five forces'' beskriver den konkurrencekontekst som virksomheden indgår i.
 Følgende beskrivelse er baseret på \citet[p.~16]{rose2012software}.

 Ifølge Porter er succesen af et startup delvis defineret af hvordan virksomheden konkurrerer med andre virksomheder på samme marked.
 Porters model kan ses på \cref{fiveforces}.

\begin{figure}[H]
	\includegraphics[width=\textwidth]{fiveforces.PNG}
	\caption{Five forces modellen}
	\label{fiveforces}
\end{figure}

\paragraph{Five Forces i SharePass' marked}
\label{par:five_forces_i_sharepass_marked}

For at beskrive det marked SharePass skal konkurrere på benyttes Five Forces modellen. 
Hver af de fem markedskræfter er beskrevet med de mest karakteriserende elementer:

\begin{itemize}
	\item \textbf{Rivalry amongst existing firms:} Der findes mange password managers, både som rent software, hardware og en kombination af de to. Typisk benyttes en abonnementsmodel.
	\item \textbf{Threat of new entrants:} Vores produkt har en væsentlig ``switching cost''\footnote{Det engelske udtryk ``switching cost'' er brugt i denne kontekst til at beskrive det en kunde skal ændre for at skifte fra et produkt til et andet.}  i forhold til eksisterende løsninger (der dog ikke er lige så integreret i virksomheden). 
	\item \textbf{Threat of substitute products:} Vores produkt kan få konkurrence af mindre avancerede løsninger der har lavere ``switching cost''.
	\item \textbf{Determinants of Supplier Power:} Vores hardware tager ikke udgangspunkt i en specifik leverandør og er derfor let at skifte. 
	\item \textbf{Determinants of Buyer Power:} Fokus er at skræddersy til virksomheder hvilket betyder at virksomhederne har en stor magt.
\end{itemize}


\section{Business model}
\begin{figure}
  \includegraphics[angle=90, height=0.95\textheight]{graphics/BM.pdf}
  \caption{Business Model Canvas.}
  \label{bm}
\end{figure}

Nu hvor idéen er afklaret og vi har undersøgt hvordan vi vil skabe værdi kan vi begynde at designe en business model.
Vi har brugt Business Model Canvas(BMC) fra \citet{osterwalder2009business} til at komme frem til vores Business Model.
BMC kan bruges til at beskrive, designe, diskutere, udfordre, forbedre og udvikle en BM.
Det er derfor rigtig godt at bruge når man er et team med mange forskellige idéer.
Her giver BMC et rigtig godt grundlag til at få bearbejdet de forskellige idéer.


BMC består af ni elementer som sørger for at man husker alle aspekter der er vigtige når man prøver at finde på eller forbedre en BM: \textit{Customer Segments, Value Propositions, Channels, Customer Relationships, Revenue Streams, Key Resources, Key Activities, Key Partnerships, Cost Structure}.
Vores første BM kan ses på \cref{bm}.
	
\subsection{Discussion}
%!TEX root = ../master.tex
\subsubsection{Karakteriser og diskuter value proposition}

Denne sektion vil diskutere den value proposition der blev præsenteret i business modellen på \cref{bm} i et bredt perspektiv.
Diskussionen vil tage udgangspunkt i koncepterne præsenteret i \citet[p.~40]{rose2012software}.

\paragraph{Type af entreprenør}
Giarratanas tre typer af entreprenør er beskrevet som følger

\begin{description}
	\item[Innovator] Skaber nye produkter der ikke er set før
	\item [Arbitrageur] Udnytter ineffektivitet i markedet. For eksempel kan nuværende løsninger være for dyre eller for besværlige for en bestemt kundegruppe
	\item [Coordinator] Udnytter ressourcer på en alternativ måde
\end{description}

Af disse typer er vores virksomhed en arbitrageur. \stefan{enig? Er vi måske en kombi mellem inno og arbi?} 
Vi bringer noget nyt til markedet ved at kombinere kendte teknologier (password manager, hardware nøgler)på en måde der eliminerer ineffektive arbejdsgange i virksomheder der muligvis allerede bruger de produkter vi kombinerer.

\paragraph{Competitor orientation}
Ifølge Mueller \citep[p.~40]{rose2012software} er der en positiv korrelation mellem succesfulde startups og startups der har en retning i forhold til konkurrencen (competitor orientation).
Der findes følgende strategier:
\begin{description}
	\item[Leader] Image, innovation, kobler teknologi men brugerbehov
	\item[Follower] Optimerer features på eksisterende produkter, design
	\item [Exploiter] Optimerer kvalitet, modifikationer af eksisterende produkter
	\item [Extender] Prisoptimering, fokus på salg
\end{description}

Med disse hører følgende business strategier:
\begin{description}
	\item[Cost leadership] Billigere produkter en konkurrencen
	\item [Differentiation] Produkter der er unikke og skiller sig ud
	\item [Niche] Konkurrerer i et område med få konkurrenter
	\item [Speed to market] Evnen til at innovere og sende nyskabelser på markedet hurtigst.
\end{description}

Vores virksomhed forsøger at være leader indenfor den del af markedet vi åbner for.
Det er derfor vigtigt at vi fokuserer på at have et godt image.
Dette vil vi sørge for ved at tilbyde direkte kundesupport og udvikle et sikkert og brugervenligt system.

Af de fire business strategier er vore virksomhed i en niche hvor vi konkurrerer med nogle få specialiserede konkurrenter.
Det er vigtigt for virkomheden at have vægt på differentiation, da vi er nødt til at skille os ud fra de etablerede virksomheder for at overbevise kunder om at vores løsning, der typisk vil være mere omfattende end konkurrentens, er værd at implementere.

\paragraph{Entry and growth}
Ifølge Porter er der altid "barriers of entry" når man begiver sig ind på et allerede eksisterende marked \citep[p.~50]{rose2012software}.
Naturligvis vil eksisterende virksomheder gøre alt for at man som ny på markedet ikke tager deres kunder.
Desuden har man som startup mange ulemper der giver de etablerede virksomheder fordelen: få midler, få ansatte med mange opgaver, intet omdømme osv.
Ojala og Tyrväinen har opsat følgende drivkræfter for et stattups succes i at begive sig ind på markedet: \citep[p.~50]{rose2012software}

\begin{description}
	\item[Innovation] Software produktet, algoritmer og teknikker
	\item [Capabilities of entrepreneurs] Både specialiserede tekniske evner og evner som entreprenør
	\item [Patents] Beskyttelse af intellektuel ejendom
	\item [Specialization in a specific market niche] 
\end{description}

For vores virksomhed er det vigtigt at have vægt på alle fire drivkræfter, men især fokus på innovation af produktet og specialisering i nichen er vigtigt for at det skal blive en succes.
Det er afgørende at vi skiller os ud fra de etablerede virksomheder og udvikler denne differentiering til vores fordel.
\paragraph{How does our BM support our implementation?}
\bruno{IVAN? Hvad skal der være her? Vi har jo lavet det ud fra at vi kan implementere vores produkt.}
\ivan{Tænk på big-bang, MVP og bootstrapping.}
\paragraph{How does the canvas model change over time?}
Lige nu er der meget fokus på sikkerhed så det er godt at hoppe ind på markedet nu - det kan være at dette fokus ændre sig.
Det ligger op til at undersøge nye markeder.
En mulighed er at udvide til det private marked, dvs. at customer segments bliver udvidet.
En anden mulighed er at tilbyde flere services:
\begin{itemize}
\item Tilbyde forskellige slags hardware løsninger til at opbevare passwords(i stedet for kun NFC).
\item Tilbyde en server til at distribuere passwords til kunder.
\end{itemize}
Hvis disse to services tilføjes ændres value proposition så vores løsning kan skræddersyes mere til kunden.
Key activities og key resources ændres da der nu skal tages højde for at udvikle og vedligeholde de nye services.
For at spare penge som virksomhed eller gøre det mere tilgængeligt for kunder med færre midler kunne man ændre after sales til at foregå over en webside.
Dette ændre revenue streams til at der tilbydes billigere support eller kun en slags support over webside.
Det ændrer også customer segments fordi det giver adgang til firmaer og private med færre midler.

\subsection{Pattern overvejelser}
\todo{Overvejelser ud fra Osterwalder side 52-119.. R I P}
\todo{Ivan - how much?}
\todo{Hvorfor er det ikke en af de 5?}

\paragraph{Design tilgang}
\stefan{Skal vi have det her med: ``You may discuss your design approach (e.g. Customer Insights or Scenarios) using relevant concepts from the course notes''}

\paragraph{Strategy concerns}
\bruno{Læs strategy oste-bogen: 192-231}

\section{Analyse, design and enact(ADE) eller consider, do and adjust(CDA)}
Følgende er baseret på \cite[pp. 27-38]{rose2012software}.

I følge Sarasvathy er der to paradigmer indenfor entrepenørskab.
Disse gør sig også gældende indenfor software entrepenørskab.
De to paradigmer, ADE og CDA, agerer poler, på samme måde som traditionel og agil software udvikling.

Det første af de to paradigmer er ADE:\\
\begin{center}
\includegraphics[width=.5\textwidth]{graphics/ade}
\end{center}

ADE minder i høj grad om traditionel software udvikling, idet der forsøges at planlægges og designes således der vil forekomme færre uforudsete problemer.

I modsætning til ADE vil CDA ikke forudsige, men nærmere forsøge at kontrollere.
Dette er en iterativ proces, i stil med agil software udvikling:\\
\begin{center}
\includegraphics[width=.5\textwidth]{graphics/cda}
\end{center}

Som \citet{rose2012software} nævner vil de fleste organisationer falde under begge paradigmer.
Dette ser vi umiddelbart også som værende tilfældet for vores forretning, hvorfor vi systematisk vil gennemgå sammenligningstabellen i \citet[p. 38]{rose2012software}.

\paragraph{Process:} I første omgang vil der vælges en sekventiel tilgang, da vi har hardware og software der skal laves og begge disse er meget sikkerhedskritiske.
Det menes derfor at det kunne være kritisk at få et ufærdigt produkt ud, da tilliden til vores forretning er vigtig.
Efter første udrulning vil der derimod kunne indføres en mere iterativ tilgang, når først vi har vundet den tillid.

\paragraph{Understanding of future:} Vi vil gerne styre fremtiden, altså sætte en høj standard for sikkerhed som andre vil få svært ved at følge.
På denne måde skal vi være gode til at tilpasse os, så vores produkt følger med til de stigende/ændrede generelle sikkerhedskrav.

\paragraph{Attitude to market:} Vores produkt er som sådan ikke nyt, da der allerede findes adskillige måder at håndtere passwords og deling af passwords.
Vi forsøger derimod at skabe et nyt marked, ved at bringe et kombineret sikkerhedsprodukt til virksomheder, hvor de måske ikke er klar over at de har brug for det.

\paragraph{Attitude to technology development:} Vi vil gerne være foran de andre, så vi kan sætte standarden for hvad god sikkerhed og så vi er de eneste der kan levere det optimalt.

\paragraph{Role of business planning:} Vi har ikke råd til fejl her, da hvis vi først får leveret en dårlig løsning, vil vores (potentielle) kunder få svært ved at have tillid til os efterfølgende.
Derfor vil vi gerne ramme vores segment meget præcist, således at vi kan levere det bedst mulige produkt.

\paragraph{Software development style:} Agile og lean, da vi gerne vil have et velafprøvet og veltestet produkt.
Usability er igen vigtigt, da vi gerne vil have vores produkt så seamless integreret som muligt.

\paragraph{Attitude to change:} Vi prøver at skabe et nyt marked, så vi ser ændringer som muligheder for at lave et bedre og mere fleksibelt produkt.

\paragraph{Funding approach:} Venture capital, så vi kan få sikret kvalitet inden produktet frigives, sådan at vores kunder ikke mister tilliden til os, ved en for dårlig tidlig løsning.

\paragraph{Approach to others working in the same areas:} Samarbejde er godt.
Hvis vi kan undgå at opfinde/lave alting selv er det en fordel.
Så kan vi udnytte andres erfaringer, sammenflette idéer og sælge dem videre som pakkeløsning.
Vores produkt er som sagt ikke nyskabende, vi prøver blot at samle tidligere løsninger og sælge dem til et nyt marked.

\paragraph{Approach to intellectual property:} Platformen er lukket, men de enkelte dele er åbne.
Vi prøver ikke at opfinde noget nyt, men prøver derimod at skabe en platform til bedre at udnytte de eksisterende teknologier i det nyskabte marked.

\paragraph{Partnering and Networking:} Netværk og samarbejde er godt, samme som ovenover.

\paragraph{Time to market:} Se `Funding approach'.

\subsection{Phases or activities in the business model process}

\subparagraph{Mobilize}
\begin{itemize}
\item Test preliminary business ideas (Diskuter internt, cross functional/tier)
\item Assemble team (Relater Resource-based (husk usability-gut) og value-chain)
\end{itemize}

\subparagraph{Understanding}
\begin{itemize}
\item Study potential customers (Undersøg nuværende arbejdsprocesser og sikkerhedsprotokoller, hvordan de håndterer passwords, hos potentielle kunders virksomhed)
\item Interview experts (Interview René Rydhof, Kramshøj, Jacob Nielsen)
\item Research already tried (Nuværende relaterede firmaer og deres produkter)
\end{itemize}

\subparagraph{Design}
\begin{itemize}
\item Prototype setup inhouse (2 rum med kort-indgang og computere hvor passwords kan bruges)
\item Brainstormet og testet BMs (Venture Capital og Ivan)
\end{itemize}

\subparagraph{Implement}
\begin{itemize}
  \item Communicate and involve (Alle i vores virksomhed skal være med på den)
\end{itemize}

\subparagraph{Manage}
\begin{itemize}
  \item Se ovenover: Attitude to change, technology development, understanding future
\end{itemize}

\todo{Oste-bogen, side 244-?}


\paragraph{Financial concerns relativ to the choice of paradigm}
\bruno{On hold}

\section{Diskuter to papers}
\subsection{Er vores valg a paradigme teoretisk eller empirisk validt?}
\subsection{Relater concepter brugt til artiklerne fra kurset}

\printbibliography[heading=bibintoc]

\end{document}