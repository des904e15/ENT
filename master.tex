\documentclass[a4paper]{article}

\usepackage[english]{babel}
\usepackage[utf8]{inputenc}
\usepackage{amsmath}
\usepackage{graphicx}
\usepackage[colorinlistoftodos]{todonotes}

\title{Your Paper}

\author{You}

\date{\today}

\begin{document}
\maketitle

\section{Motivation}
\subsection{Storytelling}
\section{Idé}
\todo{Valgte teorier: Value chain, five forces, resources-based theory}

Password manager med deling for firmaer og private

En form for hardware der lagrer krypterede passwords fx en USB nøgle med indbygget software der sørger for at anvende de rigtige koder.
Lokal central der indeholder alle passwords, som nøgle synkroniseres fra.
Administrering af egne nøgler, samt deling af disse.
Automatisk opdatering af intra-net systemers nøgler (tænk nyt password hver måned).

\subsection{Value chain}
Taget fra \cite[p.~12]{rose2012software}.
\subsubsection*{Primary activities}
BUY:
\begin{itemize}
\item NFC kort
\item Fysisk login med NFC kort hos firma - her opdateres passwords på kortet
\end{itemize}
ADD VALUE:
\begin{itemize}
\item Sikkerhedsprotokol mellem fysisk login portal og server
\item Sikkerhedsprotokol mellem klient og NFC kort
\item Sikkerhedsprotokol mellem fysisk login portal og NFC kort
\item Software til password server
\end{itemize}
SELL:
\begin{itemize}
\item Sælg bedste produkt i verden
\item Tjen en masse penge på support
\end{itemize}

\subsubsection*{Support activities}
SELL
\begin{itemize}
\item Marketing
\item Hyre - sikkerhedsekspert, hardwaredesigner
\end{itemize}

\section{Business model}
\todo{Se på requirements}

\end{document}
