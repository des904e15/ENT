\documentclass[a4paper]{article}

\usepackage[english]{babel}
\usepackage[utf8]{inputenc}
\usepackage{amsmath}
\usepackage{graphicx}
\usepackage[colorinlistoftodos]{todonotes}
\usepackage[backend=bibtex,
  bibencoding=utf8,natbib
  ]{biblatex}
\addbibresource{bib}

\title{Your Paper}

\author{You}

\date{\today}

\begin{document}
\maketitle

\section{Motivation}
\subsection{Storytelling - en comic strip}

\section{Idé}
\todo{Overvej fysisk login portal!}
\todo{Valgte teorier: Value chain, five forces, resources-based theory}

Password manager med deling for firmaer og private

En form for hardware der lagrer krypterede passwords fx en USB nøgle med indbygget software der sørger for at anvende de rigtige koder.
Lokal central der indeholder alle passwords, som nøgle synkroniseres fra.
Administrering af egne nøgler, samt deling af disse.
Automatisk opdatering af intra-net systemers nøgler (tænk nyt password hver måned).

\subsection{Value chain}
Taget fra \citet[p.~12]{rose2012software}.
\subsubsection*{Primary activities}
BUY:
\begin{itemize}
\item NFC kort
\item Fysisk login med NFC kort hos firma - her opdateres passwords på kortet
\end{itemize}
ADD VALUE:
\begin{itemize}
\item Sikkerhedsprotokol mellem fysisk login portal og server
\item Sikkerhedsprotokol mellem klient og NFC kort
\item Sikkerhedsprotokol mellem fysisk login portal og NFC kort
\item Software til password server
\end{itemize}
SELL:
\begin{itemize}
\item Sælg bedste produkt i verden
\item Tjen en masse penge på support
\end{itemize}

\subsubsection*{Support activities}
SELL
\begin{itemize}
\item Marketing
\item Hyre - sikkerhedsekspert, hardwaredesigner
\end{itemize}

\subsection{Resource-based theory}
Taget fra \citet[p.~12]{rose2012software}.
\paragraph{Ressourcer:}
\begin{itemize}
\item Ansatte:
  \begin{itemize}
  \item Programmør
  \item Sikkerhedsekspert
  \item Hardwaredesigner
  \item Sælger
  \item Systemarkitekt
  \end{itemize}
\item Produkter
  \begin{itemize}
  \item NFC kort
  \item Fysisk login portal
  \item Software klient - ansattes pc kommunikation med NFC kort
  \item Server sofware - deling og håndtering af passwords og licenser
  \item Sikkerhedsprotokol mellem fysisk login portal og server
  \item Sikkerhedsprotokol mellem klient og NFC kort
  \item Sikkerhedsprotokol mellem fysisk login portal og NFC kort
  \end{itemize}
\item Pakkeløsning
\item Kundeservice
  \begin{itemize}
  \item Opsætning
  \item Udvidelse
  \item Support
  \end{itemize}
\item Viden om:
  \begin{itemize}
  \item Sikkerhed
  \item Kunders(virksomheder) workflow
  \item Softwarearkitektur
  \end{itemize}
\end{itemize}
\paragraph{Formåen:}
Levere en samlet løsning, skræddersyet til kunden som er sikker at bruge.
En sikker løsning for kunden og hans ansatte.
De ansatte skal ikke tænke på sikkerhed men data er alligevel beskyttet af sikre passwords.

\section{Business model}
\todo{Se på requirements}

\printbibliography[heading=bibintoc]

\end{document}
